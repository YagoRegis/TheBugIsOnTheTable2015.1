%Compilação: 
%pdflatex -shell-escape PDS_Robotica.tex

\documentclass[12pt,a4paper]{article}

\usepackage{hyperref} %linkar sumário com a página%
\hypersetup{
	colorlinks = true,
	linkcolor = black
}
\usepackage{amsmath} %remover numeracao das formulas%
\usepackage{graphicx} %imagens%
\usepackage{booktabs} %tabela mais bonita%
\usepackage{minted} %code highlitning%
\usepackage{eso-pic}
\usepackage[brazilian]{babel} %hifenização%
\usepackage[utf8]{inputenc} %palavras acentuadas%
\usepackage[T1]{fontenc} %codificação da fonte%

\title{PDS - Processo de Desenvolvimento de Software\\The Bug is on the Table}
\date{03-07-2015}
\author{Bruno Rodrigues\\Iago Rodrigues\\Jonathan Rufino\\Matheus Andrade\\Yago 
Regis}

\begin{document}

\maketitle
\pagenumbering{gobble}

\newpage
\pagenumbering{arabic}

\newpage
\listoffigures

\newpage
\listoftables

\newpage
\tableofcontents

\newpage
\section{Missões}
	\paragraph{}
		O documento do desafio Fúria da Natureza[1], nos da diretrizes à 
		respeito de todas as missões que podem ser realizadas. Cada uma possui 
		sua descrição, condições finais e a respectiva pontuação.

	\subsection{Ordem de Pontuação das Missões}
		\paragraph{}
			As missões foram listadas em ordem decrescente de pontuação, de 
			forma que a pontuação considerada seja a maior possível de cada uma.

		\begin{table}[h!]
			\begin{center}
				\begin{tabular}{cc}
					\toprule
						Pontuação & Missão\\
					\midrule
						66 & Família\\
						31 & Obstáculos\\
						30 & Galho da Árvore\\
						30 & Pista de Pouso\\
						30 & Sinal de Evacuação\\
						30 & Teste de Isolamento de Base\\
						30 & Avião de Carga\\
						25 & Ambulância\\
						25 & Casa Elevada\\
						25 & Construção de Código\\
						25 & Zona de Segurança\\
						20 & Camiñhão de Suplementos\\
						20 & Relocação de Construção\\
						20 & Tsunami\\
						18 & Segurança\\
						15 & Água\\
						15 & Animais\\
						4 & Suplementos e Equipamentos\\
					\bottomrule
				\end{tabular}
				\caption{Relação de pontos por missão.}
				\label{tab:table_pontuation}
			\end{center}
		\end{table}

	\subsection{Ordem de Dificuldade das Missões}
		\paragraph{}
			Lista de missões ordenadas por dificuldade, das mais fáceis para as 
			mais difíceis, onde cada membro do grupo classificou individualmente 
			cada missão e o resultado de todos foi mesclado nesta única tabela.
		\paragraph{}
			A partir daqui foi feito um cruzamento entre a pontuação e a 
			dificuldade de cada missão para priorização das mesmas, e assim 
			decidir a ordem em que o robô executará elas, para obter assim maior 
			quantidade de pontos no menor espaço de tempo possível, ou seja, os 
			2:30 minutos.

		\begin{table}[h!]
			\begin{center}
				\begin{tabular}{cc}
					\toprule
						Dificuldade & Missão\\
					\midrule
						FÁCIL & Zona de Segurança\\
					    FÁCIL & Caminhão de Suplementos\\
					    FÁCIL & Tsunami\\
					    MÉDIO & Família\\
					    MÉDIO & Galho da Árvore\\
					    MÉDIO & Pista de Pouso\\
					    MÉDIO & Avião de Carga\\
					    MÉDIO & Água\\
					    MÉDIO & Animais\\
					    MÉDIO & Suplementos e Equipamentos\\
						DIFÍCIL & Obstáculos\\
					    DIFÍCIL & Teste de Isolamento da Base\\
					    DIFÍCIL & Casa Elevada\\
					    DIFÍCIL & Construção de Código\\
					    DIFÍCIL & Relocação de Construção\\
					    DIFÍCIL & Segurança\\
				    \bottomrule
				\end{tabular}
				\caption{Relação de dificuldade das missões}
				\label{tab:table_dificulty}
			\end{center}
		\end{table}

	\subsection{Definindo o Problema}
		\paragraph{}
			Algumas missões são completamente independentes uma das outras, 
			entretanto, existem missões que devem seguir uma sequência lógica 
			para que não atrapalhem as demais. Cada missão deve possuir um 
			planejamento antes de ser desenvolvida e executada.

\newpage
\section{API}
	\subsection{Objetivo}
		\paragraph{}
			O objetivo da API (Application Programming Interface) é 
			disponibilizar um conjunto de rotinas e padrões que serão utilizados 
			em um ou mais Softwares em que os detalhes de sua implementação não 
			são tão importantes, mas a boa execução de seus serviços sim.
		\paragraph{}
			Para o caso deste projeto, uma API foi implementada visando separar 
			as funções básicas que são mais utilizadas pelo robô para executar 
			as missões. As rotinas identificadas foram:

		\begin{itemize}
			\item Virar – Foi observada a necessidade do robô fazer curvas 
			durante diversas missões;
			\item Mover – É a operação mais básica realizada pelo robô;
			\item Controle de Garra – Outra função bastante utilizada em 
			diversas missões;
		\end{itemize}

		\paragraph{}
			Além disto, todas estas funções abstraíram características da 
			linguagem NXC, que trabalha muito com graus, tempo (em segundos) e 
			curvas de apenas um motor, para executar suas próprias rotinas e 
			adaptou-as para medições mais utilizadas, como se mover em 
			centímetros, e fazer curvas em graus (já considerando ambos os 
			motores).

	\subsection{Código NXC}
		\inputminted[linenos, frame = single]{c}{../theBugAPI.c}
			\label{lst: thebugapi.c}
		
		\inputminted[linenos, frame = single]{c}{../theBugAPI.h}
			\label{lst: thebugapi.h}

\newpage
\section{Caminhão de Suplementos, Ambulância, Sinal de Evacuação e Isolamento de 
Construção}

	\subsection{Objetivo da missão}
		\begin{itemize}
			\item O caminhão de suplementos está tocando o tapete na região 
			amarela;
			\item A ambulância está na área amarela;		
			\item Todas as rodas da ambulância estão tocando o tapete;
			\item O sinal está obviamente em pé (não precisa ser na vertical), 
			mantido no lugar apenas pelo atrito da viga com o tapete;
			\item Nenhum parte do modelo de missão está sendo tocado pelo robô 
			ou qualquer obstáculo estratégico;
			\item A ambulância está na área amarela;
			Todas as rodas da ambulância estão tocando o tapete.
			\item O prédio oeste está intacto: seus 4 segmentos estão a 90* do 
			tapete, e “perfeitamente” alinhados.
			\item O edifício leste está obviamente danificado.
			\item *Nada está tocando nenhum dos prédios exceto a base de 
			rolamento.
			\item *Nada nunca tocou nenhum dos prédios exceto a base de 
			rolamento.
			\item O dano foi causado unicamente pelo movimento da base de 
			rolamento.
			\item (*Exceção: Segmentos caídos do edifício leste podem tocar o 
			tapete e/ou o edifício oeste
			por acaso.)
		\end{itemize}

	\subsection{Passos envolvidos}
		Sair da base\\
		Pegar o caminhão\\
		Pegar a ambulância\\
		Sseguir até a área amarela\\
		Empurrar o Sinal de Evacuação\\
		Destruir o prédio direto das torres\\
		Voltar para a base\\

	\subsection{Pseudo código}
		\begin{enumerate}
			\item Mova em frente 38cm
		    \item Vire à direita 85 graus
		    \item Mova em frente 50cm, pegue o caminhão
			\item Vire à esquerda 9 graus
			\item Mova em frente 20cm, pegue a ambulância
			\item Mova em frente 100cm, para chegar na área azul
		    \item Vire à esquerda 80 graus
			\item Mova em frente 25cm
			\item Mova para trás 20cm
		    \item Vire à esquerda 30 graus
		    \item Mova para trás 40cm
		    \item Vire à direita 35 graus
		    \item Mova em frente 21cm, para empurrar a placa
		    \item Mova para traś 30cm
		    \item Vire à esquerda 90 graus
		    \item Mova em frente 44cm
		    \item Mova para trás 15cm
		    \item Vire à esquerda 83 graus
		    \item Mova em frente 150cm
		    \item Encerre o programa
		\end{enumerate}

	\subsection{Código NXC}
		\inputminted[linenos, frame = single]{c}{../Ambulancia.nxc}

\newpage
\section{Galho da Árvore, Animais e Equipamentos}

	\subsection{Objetivo da Missão}
		\begin{itemize}
			\item O galho leste da árvore está mais próximo do tapete do que os 
			cabos elétricos estão;
			\item A árvore e o modelo de missão dos cabos elétricos estão para 
			cima, retos, tocando o tapete;
			\item Ao menos um animal está com pelo menos uma pessoa em uma 
			região colorida;
			\item Ao menos um elemento que não é água está numa região colorida 
			vermelha ou amarela (12 elementos possíveis: walkie talkie, bateria, 
			gerador, 2 combustíveis, grão, pão, remédio, rádio, lanterna, 
			motocicleta e capacete).
		\end{itemize}

	\subsection{Passos envolvidos}
		Sair da Base\\
    	Ir ao lado do galho da árvore\\
    	Derrubar o galho da árvore\\
    	Pegar os animais e equipamentos da região\\
   		Retornar a Base\\

   	\subsection{Pseudo-código}
   		\begin{enumerate}
	   		\item Mova em frente 70 cm
	   		\item Levante a garra 100 graus
	   		\item Desça a garra 100 graus
	   		\item Mova para trás 30cm
	   		\item Vire à direita 85 graus
	   		\item Mova para trás 22cm
	   		\item Vire à esquerda 90 graus
	   		\item Mova em frente 39cm
	   		\item Levante a garra 70 graus
	   		\item Mova para trás 70cm
		    \item Encerre o programa
		\end{enumerate}

	\subsection{Código NXC}
		\inputminted[linenos, frame = single]{c}{../GalhoArvore.nxc}

\newpage
\section{Tsunami}
	\subsection{Objetivo da missão}
		\begin{itemize}
			\item O galho leste da árvore está mais próximo do tapete do que os 
			cabos elétricos estão.
			\item A árvore e o modelo de missão dos cabos elétricos estão para 
			cima, retos, tocando o tapete.
		\end{itemize}

	\subsection{Passos envolvidos}
		Sair da base\\
		Ir até a estrutura do tsunami\\
		Acionar o mecanismo\\
		Retornar a base\\

	\subsection{Pseudo-código}
		\begin{enumerate}
			\item Ligue o sensor Ultrasônico
			\item Mova em frente até que o sensor leia uma distância de 23cm ou 
			menos
			\item Levante a garra 60 graus
			\item Mova para trás 70cm
			\item Encerre o programa
		\end{enumerate}

	\subsection{Código NXC}
		\inputminted[linenos, frame = single]{c}{../Ondas.nxc}

\newpage
\section{Família, Água, Segurança, Animais, Suplementos e Equipamentos, Zona de 
Segurança}
	\subsection{Objetivo da missão}
		\begin{itemize}
			\item Ao menos duas pessoas estão juntas em uma área colorida.
			\item Ao menos uma pessoa está com uma água (engarrafada) na mesma 
			região.
			\item Ao menos uma pessoa está na região colorida vermelha ou 
			amarela.
			\item Ao menos um animal está com pelo menos uma pessoa em uma 
			região colorida.
			\item Ao menos um elemento que não é água está numa região colorida 
			vermelha ou amarela.
			\item O robô está na região vermelha no final da partida.
		\end{itemize}

	\subsection{Passos envolvidos}
		Sair da base\\
		Pegar a pessoa ao lado da casa\\
		Ir para a região vermelha\\

	\subsection{Pseudo-código}
		\begin{enumerate}
			\item Mova em frente 43cm
			\item Vire à direita 85 graus
			\item Mova em frente 67cm
			\item Vire à esquerda 78 graus
			\item Mova em frente 22cm
			\item Levante a garra 55 graus
			\item Mova para trás 15cm
			\item Vire à direita 85 graus
			\item Mova em frente 80cm
			\item Vire à direita 44cm
			\item Ligue o sensor RGB
			\item Mova em frente até o sensor detectar a cor vermelha
			\item Encerre o programa
		\end{enumerate}

	\subsection{Código NXC}
		\inputminted[linenos, frame = single]{c}{../ZonaDeSeguranca.nxc}

\newpage
\section{Musica - Game of Thrones Main Theme}
	\subsection{Objetivo}
		\paragraph{}
			Uma outra característica importante presente nos robõs Mindstorms é 
			a capacidade de reproduzir sons ou músicas completas. Desta forma, o 
			grupo propôs a execução de uma música pelo modelo de missão, no 
			intuito de receber a bonificação de 5 pontos. A música escolhida foi
			o tema de abertura da série de televisão Game of Thrones.
		\paragraph{}
			Apenas a primeira parte foi implementada, visto que a música em 
			seguida fica mais complexa e o grupo conseguia fazer o robô executar 
			apenas uma nota por vez, e que, portanto, iria perder a qualidade 
			da música.
		\paragraph{}
			Foram criadas constantes baseada nas figuras musicais encontradas na 
			partitura e o tempo indicado na partitura para facilitar a 
			implementação. Foi criado constante para a primeira oitava que o 
			robô consegue tocar, para que pudesse servir de apoio para 
			implementação de funções que encontravam as frequências das notas 
			sem precisar escrever elas diretamente na função “PlayToneEx”, 
			contudo, essa última parte não foi implementada.

	\subsection{Código NXC}
		\inputminted[linenos, frame = single]{c}{../GameOfThrones.nxc}

\newpage
\section{Robô}
	\paragraph{}
		Para a montagem do robô foi utilizado o modelo 3-Motor Chassis[3], e 
		algumas adaptações foram realizadas afim de melhor atender as 
		necessidades identificadas pelo grupo.

	\begin{figure}[H]
		\includegraphics[width=\linewidth]{../Images/robot_1.JPG}
		\caption{Visão geral do robô.}
		\label{fig:robot_1}
	\end{figure}

	\begin{figure}[H]
		\includegraphics[width=\linewidth]{../Images/robot_2.JPG}
		\caption{Vista frontal.}
		\label{fig:robot_2}
	\end{figure}

	\begin{figure}[H]
		\includegraphics[width=\linewidth]{../Images/robot_3.JPG}
		\caption{Vista lateral.}
		\label{fig:robot_3}
	\end{figure}

	\begin{figure}[H]
		\includegraphics[width=\linewidth]{../Images/robot_4.JPG}
		\caption{Vista inferior.}
		\label{fig:robot_4}
	\end{figure}

\newpage
\section{Garras}
	\paragraph{}
		Para que as missões fossem efetivamente cumpridas, três garras 
		removíveis foram montadas. O objetivo foi trocar de garra para um grupo 
		de missões específicas, afim de que as mesmas pudessem ser realizadas da 
		melhor forma possível.
	\paragraph{}
		A Figura \ref{fig:claw_1} mostra a primeira garra que foi construída. O 
		objetivo desta é agrupar a ambulância e o caminhão de suplementos dentro 
		da garra e levá-los até a área pretendida.

	\begin{figure}[H]
		\includegraphics[width=\linewidth]{../Images/claw_1.JPG}
		\caption{Base para acoplamento dos veículos.}
		\label{fig:claw_1}
	\end{figure}

	\paragraph{}
		A Figura \ref{fig:claw_2} mostra a garra construída com o objetivo de se 
		levantar para derrubar o galho da árvore e capturar objetos para 
		levá-los em áreas distintas. Ela também possui um sensor RGB com 
		finalidade de perceber as cores do tapete enviando um sinal para o 
		microcontrolador que irá parar o robô caso receba um sinal de cor 
		vermelha.

	\begin{figure}[H]
		\includegraphics[width=\linewidth]{../Images/claw_2.JPG}
		\caption{Garra para derrubar o galho e capturar objetos.}
		\label{fig:claw_2}
	\end{figure}

	\paragraph{}
		A Figura \ref{fig:claw_3} mostra a garra que tem como objetivo único de 
		se levantar para acionar o mecanismo que derruba as “ondas”.	

	\begin{figure}[H]
		\includegraphics[width=\linewidth]{../Images/claw_3.JPG}
		\caption{Braço para acionamento do mecanismo das ondas.}
		\label{fig:claw_3}
	\end{figure}

\newpage
\section{Referências}
	FIRST \& LEGO. Nature's Fury. Disponível em: 
	<http://www.firstlegoleagu\\e.org/challenge/2013naturesfury>.\\\\
	NXTPROGRAMS. 3-Motor Chassis. Disponível em: 
	<http://www.nxtprogr\\ams.com/NXT2/3-motor\_chassis/steps.html>.\\\\
	SHEA. Programming NXT color sensor with NXC?. Disponível em: 
	<http://\\bricks.stackexchange.com/questions/6176/programming-nxt-color-
	sensor-wi\\th-nxc>.

\end{document}